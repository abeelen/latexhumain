\phantomsection\addcontentsline{toc}{section}{Au sujet de ce livre}\section*{Au sujet de ce livre}\thispagestyle{plain}

Il va de soi que ce livre a été composé avec \XeLaTeX. Outre les packages dont il traite, j'ai utilisé le package  \package{minted} pour les citations de code ; les packages \package{mdframed} et \package{framed} pour les boîtes colorées.



Ce livre est diffusé sous licence \emph{Creatives Commons - Paternité - Partage des Conditions Initiales à l'Identique 3.0 France}. Sommairement\footnote{Pour les détails, je renvoie au texte intégral de la licence : \url{http://creativecommons.org/licenses/by-sa/3.0/fr/legalcode}.}, cela signifie que vous pouvez le diffuser, le dupliquer, le publier et même le modifier si vous respectez deux conditions:
\begin{enumerate}
\item que vous citiez mon nom\footnote{Et que vous ne portiez  pas atteinte à mes droits moraux.};
\item que vous offriez les mêmes droits aux destinataires de vos diffusions\footnote{Les images de pas et d'éclair servant à indiquer les encarts sont tirées du domaine public et (légèrement) modifiées par mes soins. Elles ne sont donc pas affectées par ces règles. Voir \url{http://www.openclipart.org/detail/154855/green-steps-by-netalloy} et \url{http://thenounproject.com/noun/high-voltage/}. L'image de couverture est de Duane  Bibby, avec une légère modification. Voir \url{http://www.ctan.org/lion.html}.}.
\end{enumerate}

Bien sûr, si vous souhaitez me soutenir, vous pouvez acheter cet ouvrage en version papier, ou simplement m'envoyer un petit mot --- vous trouverez aisément comment me contacter sur Internet.

Si vous souhaitez améliorer cette œuvre, soyez le bienvenu. Le code est mis à disposition sur GitHub\footnote{À l'adresse \url{https://github.com/maieul/latexhumain}.}, un service fonctionnant à l'aide de l'outil de travail collaboratif Git\renvoi{svn} mais disposant d'une interface d'édition en ligne. 

N'hésitez pas à me demander un accès à l'édition du projet ! 

