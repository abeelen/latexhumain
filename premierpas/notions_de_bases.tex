\chapter[Mit XeLaTeX beginnen]{Mit \XeLaTeX{} beginnen}\label{commencer}

\begin{intro}
Wir nehmen an, dass Sie bereits \LaTeX\renvoi{install} und einen auf \LaTeX{} spezialisierten Texteditor\renvoi{logiciels} installiert haben. Siehe im Anhang.

Zunächst ist sicherzustellen, dass das Textverarbeitungsprogramm die Codiereung UTF-8 verwendet.\footnote{Die Einstellung findet sich in der Regel in den Programmeinstellungen, meist in der Rubrik \emph{Dateiformat} oder \emph{Zeichencodierung}: konsultieren Sie gegebenenfalls das Handbuch des Programms.} Wir werden später \renvoi{utf8} auf den Zweck einer solcher Codierung zurückkommen. Im Augenblick genügt es zu wissen, dass diese die Verwendung nicht-lateinischer Zeichen erlaubt.\footnote{Kyrillisch, Griechisch, Sanskrit, Hebräisch, etc. Auch die Verwendung außerirdischer Zeichen ist möglich.}
\end{intro}

\section{Ein erstes Dokument}

Geben Sie in Ihrem Editor den folgenden Code ein\footnote{Wie wir in der Einleitung erläutert haben (p.~\pageref{colorationsyntax}), hat die Kolorierung, die Sie hier sehen, wenn Sie die elektronische Version dieses Buches lesen, eine syntaktische Bedeutung. Machen Sie sich keine Sorgen, wenn sie in Ihrem Editor anders ist. Denken Sie außerdem nicht, dass Ihr kompilierter Text so aussehen wird.} und klicken Sie auf den Knopf zur Kompilation mit \XeLaTeX \footnote{Dessen Position hängt von Ihrem Texteditor ab. Im Moment können Sie sich mit diesem Knopf begnügen. Später werden Sie einige Befehle lernen müssen, doch machen Sie sich keine Sorgen. Alles wird zu erklärt.}:

\begin{attention}
Wenn Sie einen Mac verwenden, sind die für \LaTeX notwendigen Zeichen nicht auf Ihrer Tastatur sichtbar. Wir haben im Anhang angegeben, wie diese eingegeben werden können\renvoi{claviermac}.
\end{attention}
\inputminted{exemples/premierpas/notions/1.tex}

Betrachten Sie das erzeugte PDF. Um die Grundlagen von \LaTeX zu verstehen, werden wir nun zunächst den Code, den Sie kopiert haben, Zeile für Zeile lesen und kommentieren.

\section{Struktur eines \LaTeX-Dokuments}

\subsection{Die Dokumentenklasse}
Die erste Linie \csp{documentclass}\verb|[12pt,a4paper]{book}| bestimmt die Dokumentklasse, in diesem Fall \classe{book}. Eine Klasse beinhaltet eine editorische Entscheidung -- Layout und allgemeine Gliederung des Dokuments. Die Wahl der Klasse beeinflusst unter anderem:

\begin{itemize}
\item Die Anzahl der verfügbaren Überschriftsebenen.
\item Die Seitenränder.
\item Kopf- und Fußzeilen.
\end{itemize}

Es existieren standardmäßig mehrere Dokumentenklassen: etwa \classe{book}, mit der Bücher verfasst werden können; die \classe{article} eignet sich für Aufsätze; \classe{beamer} ist zur Erstellung von Präsentationen vorgesehen. Wir werden uns in dieser Einführung hauptsächlich mit den ersten beiden Klassen beschäftigen, \classe{beamer} aber auch kurz vorstellen\renvoi{beamer}.


Jedes \LaTeX-Dokument beginnt mit der Deklaration der Dokumentenklasse. Die Syntax lautet: \cs{documentclass}\oarg{Optionen}\marg{Klasse}.

\begin{attention}
In diesem Buch dienen Texte in spitzen Klammern (\arg{wie hier}) als Platzhalter und müssen in Ihrer \ext{tex}-Datei durch einen bestimmten Ausdruck ersetzt werden.
\end{attention}

Die Optionen dienen dazu, bestimmte Eigenschaften der Klasse genauer zu bestimmen. In unserem Beispiel wählen wir 12~pt als Größe der Schriftart des Fließtextes und A4 als Papierformat. Mehrere Optionen können angegeben und durch Kommas getrennt werden.\label{optionsclasse}

Hier einige verfügbare und für Geisteswissenschaften nützliche Optionen:

\begin{description}
\item[10pt] für eine Grundschriftart der Größe 10~pt.
\item[11pt] für eine Grundschriftart der Größe 11~pt.
\item[12pt] für eine Grundschriftart der Größe 12~pt.
\item[onecolumn] für einen einspaltig gesetzten Text. In den erwähnten Klassen ist das der Standard.
\item[twocolumn] für einen zweispaltig gesetzten Text.
\item[oneside] für einseitigen Druck. \label{nbsides}
\item[twoside] für doppelseitigen Druck.\footnote{Bei dieser und der vorherigen Option geht es im Wesentlichen um die Frage der Bindung. Die Option \option{twoside} erzeugt für Vorder- und Rückseiten jeweils unterschiedliche linke und rechte Ränder. Außerdem werden die Seitenzahlen abwechselnd innen und außen gedruckt.}\label{rectoverso}
\end{description}

Wir werden weitere Optionen nach und nach darstellen, nachdem die notwendigen Begriffe angesprochen wurden.

\subsection{Der Aufruf von Packages}

Betrachten wir die beiden folgenden Zeilen: 

\begin{latexcode*}{linenos,firstnumber=2}
\usepackage{fontspec}
\usepackage{polyglossia}
\end{latexcode*}

Es handelt sich, wie Sie erahnen können, um den Aufruf von Packages.\footnote{Wir haben uns bewusst entschlossen, diesen Begriff nicht zu übersetzen, um Verwechslungen vorzubeugen.} Ein Package ist eine Sammlung von Dateien, die -- einem Plugin bei Firefox vergleichbar -- weitere Funktionen zu \LaTeX hinzufügen. 

Das erste Package ist \package{fontspec}. Es ist hilfreich für fortschrittliche typographische Anwendungen, besonders um im erzeugten PDF Akzente [disposer] zu können.

Das Paket \package{fontspec} lädt automatisch \package{xunicode}, ohne dass Sie dies zusätzlich angeben müssen. Letzteres erlaubt es, unicode zu erzeugen, das auch UTF-8 genannt wird.\footnote{UTF-8 ist eigentlich nicht identisch mit Unicode, sondern stellt eine Implementierung dessen dar. Aus Gründen der Einfachheit, werden wir beide Begriffe jedoch synonym verwenden.} Daher können wir nicht-lateinische Zeichen\renvoi{utf8} in unserem \ext{tex}-Dokument verwenden.

\begin{plusloins}
Man findet im Internet häufig Hinweise, dass man zur Eingabe von \enquote{ö} den Code \verb|\"{o}| verwendet, wobei andere Sonderzeichen ähnlich eingegeben werden können. 

Dies galt früher, doch es ist bereits seit einiger Zeit nicht mehr hilfreich, diese Befehle zu lernen. Man kann Sonderzeichen ohne Bedenken \enquote{normal} eingeben. 
\end{plusloins}

Das zweite manuell geladene Package \package{polyglossia} ermöglicht, auf einfache Weise ein mehrsprachiges Dokument\renvoi{i18n} zu erzeugen, wobei die nötigen typographischen Anpassungen vorgenommen worden.

Diese drei Packages sind speziell für \XeLaTeX konzipiert: Sie funktionieren nicht mit \renvoi{TeXLaTeX} \LaTeX.

Manche Packages können mit Optionen geladen werden, die ihr Verhalten anpassen. Die Syntax lautet \csp{usepackage}\oarg{options}\marg{package}.

Im Lauf dieses Buchs werden wir verschiedene Packages [aborderons].


\begin{attention}
Im weiteren Verlauf des Buches gehen wir, wenn wir die Funktionen eines Packages beschreiben, davon aus, dass das entsprechende Package zuvor in der Präambel mit dem Befehl \csp{usepackage}\marg{package} geladen wurde.
\end{attention}

\begin{plusloins}
Wenn wir über ein Package sprechen, verweisen wir häufig auf dessen Dokumentation. Man kann diese leicht über das Terminal finden. Wie das genau funktioniert, erklären wir im Anhang \renvoi{manuels}.
\end{plusloins}

\subsection{Französisch als Hauptsprache des Dokuments\label{french}}

Direkt danach bestimmt die Zeile \csp{setmainlanguage}\verb|{french}| Französisch zur Hauptsprache des Dokuments und dass somit bei der Erstellung der Ausgabe die typographischen Gepflogenheiten des Französischen\renvoi{i18n} berücksichtigt werden sollen. Diese Zeile ist für den Compiler nur verständlich, wel wir bereits \package{polyglossia} geladen haben. 

\begin{plusloins}
Vielleicht werden Sie vom Paket \package{babel} hören. Es wird häufig anstelle von \package{polyglossia} verwendet, besonders da es älter ist. Wir haben allerdings beschlossen, uns auf \package{polyglossia} zu beschränken, da wir dieses für die Erstellungen unserer Arbeiten verwendet haben und weil es mehr Funktionen bietet, etwa für Sprachen, die nicht-lateinische Schriften verwenden.

Sie können Informationen über \package{babel} ohne Schwierigkeiten im Internet finden.
\end{plusloins}

\subsection{Der Dokumentenkörper}

Was wir bislang betrachtet haben, der Teil vor \cs{begin}\verb|{document}|, gehört zur sogenannten Präambel des Dokuments.\label{preambule} Diese Informationen erscheinen nicht im erzeugten Dokument. Es handelt sich um Metadaten, die bei dessen Erstellung berücksichtigt werden. Alle Packages, die sie verwenden möchten, müssen in der Präambel geladen werden.

Was sich zwischen den Linien  \cs{begin}\verb|{document}| und \cs{end}\verb|{document}| befindet, stellt den Körper des Dokuments dar, d.h. den eigentlichen Inhalt Ihrer Arbeit.

Nichts, was sich schließlich nach \cs{end}\verb|{document}| befindet, wird vom Compiler berücksichtigt. Sie können hier eingeben, was immer Sie wollen, weswegen wir uns nicht weiter damit befassen.

\subsection{Titel, Autor und Datum: Der Begriff des Befehls}\label{notioncommande}

\begin{latexcode*}{linenos,firstnumber=7}
\title{Un titre d'ouvrage}
\author{Le nom de son auteur}
\date{Une date}
\maketitle
\end{latexcode*}

Die ersten drei Zeilen definieren jeweils den Titel (\csp{title}), den Autor (\csp{author}) und das Datum (\csp{date}) der Arbeit. Wird das Datum nicht explizit definiert, wird das der Kompilation verwendet. Damit kein Datum erscheint, muss  \cs{date}\verb|{}| eingegeben werden.

Die letzte Zeile gibt diese Informationen aus. Wenn Sie für Ihr Dokument die Klasse \classe{book} gewählt haben, gibt der Compiler diese auf einer eigenen Seite aus. Bei der Klasse \classe{article} werden diese ohne Seitenwechsel ausgegeben.

Man kann diese Regel umgehen, indem man die Klasse mit einer speziellen Option aufruft\renvoi{optionsclasse}.
\begin{description}
\item[notitlepage] erzeugt keine spezifische Titelseite.
\item[titlepage] erzeugt eine Titelseite.
\end{description}

Wir können nun den Begriff des Befehls erklären. Ein Befehl ist ein Stück Code, der vom Compiler interpretiert wird, um verschiedene Operationen auszuführen. Es handelt sich um eine Kurzschreibweise. Hier gibt der Befehl \csp{maketitle} Informationen wie den Titel, das Datum und den Autoren der Arbeit aus, Informationen also, die der Compiler durch die zuvor verwendeten Befehle erfahren hat.

Ein Befehl kann mit (fakultativen und optionalen) Argumenten aufgerufen werden, die das Verhalten des Befehls beeinflussen.
\label{syntaxecommande}Ein Befehl wird mit folgender Syntax aufgerufen: 
\csp{nom}\oarg{opt1}\oarg{…}\oarg{optn}\marg{obl1}\marg{…}\marg{obln}.

In eckigen Klammern werden die optionalen Argumente angegeben, in geschweiften die notwendigen. Diese Argumente können selbst wiederum weitere Befehle enthalten.

Die Reihenfolge der Argumente hängt vom jeweiligen Befehl ab. Optionale Argumente sind nicht immer vor den notwendigen einzugeben: Sie können auch danach oder zwischen zwei notwendigen Argumenten eingegeben werden. Außerdem gibt es Befehle, die generell ohne Argumente aufgerufen werden, was etwa bei \cs{maketitle} der Fall ist.

\begin{attention}
Jeder öffenden eckigen oder geschweiften Klammer muss jeweils eine schließende entsprechen, sonst können beim Kompilieren Fehler entstehen.
\end{attention}

Die große Stärke von \LaTeX ist, dass durch die Verwendung von Befehlen häufig wiederkehrende Arbeitsschritte nicht immer wieder erneut ausgeführt werden müssen. Daher werden wir lernen, wie wir eigene Befehle definieren können\renvoi{creercommandes}.


\subsection{Das Verfassen des Texts}
%\subsection{Le corps du texte : la manière de rédiger}

\subsubsection{Analyse des Beispiels}
Betrachten Sie nun die folgenden Zeilen und das kompilierte Resultat.


\begin{latexcode*}{linenos,firstnumber=12}
Lorem ipsum dolor sit amet, consectetuer adipiscing elit ?
Morbi commodo ; ipsum sed pharetra gravida !
Nullam sit amet enim. Suspendisse id : velit vitae ligula.
Aliquam erat volutpat.
Sed quis velit. Nulla facilisi. Nulla libero. 

Quisque facilisis erat a dui.
Nam malesuada ornare dolor.
Cras gravida, diam sit amet rhoncus ornare, 
erat      elit consectetuer erat, id egestas pede nibh eget odio.
\end{latexcode*}


Wir können mehrere Dinge feststellen.
\begin{itemize}
\item Eine leere Zeile erzeugt einen neuen Absatz. Mehrere leere Zeilen erzeugen nur einen neuen Absatz.
\item Ein Zeilenumbruch verhält sich hingegen wie ein Leerzeichen. Das ist ein großer Unterschied zu Programmen, die nach dem WYSIWYG-Prinzip arbeiten, denn dort erzeugt ein Zeilenumbruch einen neuen Absatz.
\item Mehrere Leerzeichen erzeugen in der Ausgabe ein einzelnes Leerzeichen. 
\end{itemize}

Sie kennen nun bereits die Grundregeln, um einen Text mit \LaTeX zu formatieren.

\subsubsection{Gehen wir etwas weiter}%\subsection{Allons plus loins}

Wir sagten, dass \LaTeX einen besseren Umbruch erzeugt als WYSIWYG-Programme. Doch es ist  notwendig, es mit korrektem Code zu versorgen, damit es entscheiden kann, wie der Text gesetzt werden soll.

%\LaTeX produit automatiquement  une espace fine devant les signes de ponctuation double,\verb|!:;?| principalement, comme il se doit en bonne typographie fran\c caise\footnote{Une espace fine est une espace plus petite qu'une espace normale.}. Toutefois, nous recommandons d'insérer des espaces dans le fichier \ext{tex} avant ces signes de ponctuation double, pour le confort de lecture.

%\begin{attention}
%Les espaces avant les signes de ponctuation double sont une spécificité de la typographie française. Il ne sont généralement pas présent dans les autres langues. C'est pourquoi, si vous écrivez dans une autre langue que celle de Molière, il ne faut pas mettre ces espaces. À vous donc de choisir si vous les mettez ou non dans votre fichier source, sachant que \LaTeX les insérera pour vous le cas échéant, mais ne les supprimera pas dans les langues autres que le français.
%\end{attention}

%En revanche \emph{il est obligatoire de mettre une espace après chaque signe de ponctuation}. 
Pour ce qui est des points de suspension, il est mieux de ne pas frapper trois points à la suite, mais d'utiliser la commande \csp{ldots} qui espacera correctement les points\footnote{Il est tout à fait possible de configurer l'éditeur de texte pour qu'il remplace automatiquement trois points à la suite par cette commande.}.

En ce qui concerne les guillemets, une partie sera consacrée plus tard à l'art et la manière de faire des citations\renvoi{guillemets} en \LaTeX. Nous n'en parlons donc pas maintenant.

Prêtons attention à certaines lettres ligaturées comme  \verb|œ| et  \verb|æ|. À la différence de la plupart des traitements de texte, \LaTeX ne remplace pas automatiquement les suites \verb|oe| et \verb|ae| par \verb|œ| ou \verb|æ|. Il faut donc frapper soi-même ces caractères, ou configurer son éditeur pour qu'il effectue ce remplacement.

Signalons également trois types de tirets\label{tirets} :
\begin{enumerate}
\item \verb|-| qui produit un tiret simple (-), utilisé pour les mots composés ;
\item \verb|--| qui produit un tiret demi-cadratin (--), en théorie à utiliser pour séparer une plage de nombres ;
\item \verb|---| qui produit un tiret cadratin (---), pour des incises\footnote{Certains éditeurs préfèrent utiliser des tirets demi-cadratins.}.
\end{enumerate}
 
Enfin, il est parfois utile d'insérer une espace insécable, pour éviter que deux mots se trouvent séparés par un retour à la ligne, par exemple entre un nom de souverain et son numéro de règne : \enquote{Jean~\textsc{xxiii}}.  L'espace insécable est produit par le caractère \verb|~|.



Par ailleurs, comme vous avez pu le constater, \LaTeX interprète de manière spécifique un certain nombre de caractères : \verb|\{}~|, à quoi nous ajoutons \verb|%_&$#^|\footnote{Nous ne verrons pas l'utilité \LaTeX  de tout ces caractères, certains servant essentiellement à rédiger des formules mathématiques.}.

Comment faire si nous désirons afficher un de ces caractères ? Il faut les faire précéder du caractère~\verb|\|. Ainsi pour insérer le caractère \verb|%|, il faut écrire \verb|\%|. 

Trois exceptions toutefois :
\begin{description}
\item[\textbackslash] qui s'insère avec la commande \csp{textbackslash} ;
\item[\textasciitilde] qui s'insère avec la commande \csp{textasciitilde} ; 
\item[\textasciicircum] qui s'insère avec la commande \csp{textasciicircum}. 
\end{description}
 
\subsection{Ein Kommentar}

Die folgende Zeile ist: 
\begin{latexcode*}{linenos,firstnumber=22}
%La fin du document
\end{latexcode*}

\LaTeX kennt eine einfache Regel: Alles, was auf das Zeichen \verb|%| folgt, ist ein Kommentar.
Das bedeutet, dass es vom Compiler nicht berücksichtigt wird und daher nicht im erzeugten Dokument erscheint. 

Wir empfehlen, Kommentare zu verwenden, um die Struktur des Dokuments zu erklären und um von Ihnen erstellte Befehle zu erläutern\renvoi{creercommandes}.

Sie können sie auch verwenden, um beispielsweise Informationen zu hinterlegen, die Sie für die Übersetzung eines Textes benötigen.

Wir raten hingegen davon ab, sie für Notizen während der Überarbeitung eines Textes zu verwenden. Wir werden Ihnen später erklären, wie man einen Befehl erstellt, mit dem eigene Kommentare für die Überarbeitung mitausgegeben werden können, und einen weiteren, der sie für die Erstellung der Endfassung unterdrückt\renvoi{commentaireredac}.



\subsection{Der Begriff der Umgebung}

Wir haben bislang die Begriffe Package, Präambel und Befehl betrachtet. Es bleibt noch ein letzter zu definieren: der der Umgebung.

Eine Umgebung ist ein Teil des Dokuments mit einer spezifischen Funktion, der daher auf besondere Art und Weise behandelt wird, beispielsweise eine Liste oder ein Zitat. Wir werden verschiedene Umgebungen nach und nach kennenlernen.

Man kennzeichnet den Beginn einer Umgebung \arg{nom} mit \csp{begin}\marg{nom} und man beendet sie mit \csp{end}\marg{nom}.

In der Klasse \classe{article} existiert eine hilfreiche Umgebung: \enviro{abstract}. In dieser Umgebung kann man eine Zusammenfassung des Artikels platzieren:

\begin{latexcode}
\begin{abstract}
Hier schreiben wir eine Zusammenfassung des Artikels. 
\end{abstract}
\end{latexcode}

Es ist möglich Umgebungen zu verschachteln:

\begin{latexcode}
\begin{1}
blabla blab
\begin{2}
blabl blab
\end{2}
blabl
\end{1}
\end{latexcode}

Es ist allerdings nicht möglich, Umgebungen übereinanderzulegen: Der folgende Code funktioniert daher nicht und erzeugt einen Fehler während der Kompilation.

\begin{latexcode}
\begin{1}
blabla blab
\begin{2}
blabl blab
\end{1}
blabl
\end{2}
\end{latexcode}

\subsection{Schluss}

Sie haben nun bereits die grundlegenden Begriffe und Konzepte von \LaTeX kennengelernt. Für den Moment erscheint das sicherlich etwas verwirrend, aber im weiteren Verlauf Ihrer Lektüre werden Sie besser verstehen\ldots\footnote{Zumindest hoffen wir das!}

