\chapter{Index}

\begin{intro}

Dans ce chapitre nous verrons comment faire un ou plusieurs index\footnote{Je remercie ma sœur Enimie pour l'avoir rédigé en grande partie.}.

\end{intro}


\section{Faire un index simple avec \packagenoidx{imakeidx}}\index[pkg]{imakeidx}


\subsection{Principe de base}


Pour indexer un document, il faut utiliser le package \package{imakeidx} et placer  dans le préambule la commande \csp{makeindex}\footnote{Le \package {imakeidx} remplace à la fois le  \package{makeindex} et le  \package{splitindex}, dont il reprend les commandes. Il est plus simple à utiliser, notamment pour la mise en page de l'index \ref{styleindex}, et ne nécessite qu'une compilation, contrairement à ses prédécesseurs. Pour obtenir des informations sur les commandes d'indexation, il faut consulter les manuels de \package{makeindex} et de \package{SplitIndex}.}.

\begin{latexcode}
\usepackage{imakeidx}
\makeindex
\end{latexcode}

\subsubsection{Indexer son document}



On indexe son document avec  la commande \csp{index}\marg{entrée}\label{cmdindex}. L'entrée apparaît dans l'index sous la forme indiquée par l'argument \arg{entrée}, suivie du numéro de la page où cette commande est placée dans le texte. 

Dans l'exemple qui suit\footcite{eginhard}, l'index comporte ainsi cinq entrées: \enquote{Charlemagne}, \enquote{Adrien}, \enquote{Tassilon}, \enquote{Formose} et \enquote{Damase}:
%Changer l'exemple.

\begin{latexcode}
Tandis que Charle\index{Charlemagne} était à Rome, il convint
avec le pape Adrien\index{Adrien} qu’ils enverraient de concert
des ambassadeurs à Tassilon, duc de Bavière\index{Tassilon}
\textelp{}
Les hommes choisis et envoyés dans cette ambassade furent, de 
la part du pape, les évêques Formose\index{Formose} et 
Damase\index{Damase}\textelp{}.
\end{latexcode}



Lorsqu'une entrée est référencée deux fois dans la même page, cette page n'est indiquée qu'une seule fois. 

\begin{attention}
Il vaut mieux accoler la commande \cs{index}\marg{entrée} directement au  mot à indexer, sans laisser d'espace, pour éviter toute ambiguïté en cas de changement de page.
\end{attention}


L'indexation d'un texte n'est pas automatique. Il faut placer \cs{index}\marg{entrée} à chaque endroit  que l'on veut référencer. On  peut bien sûr créer une commande  spécifique pour combiner l'indexation avec d'autres actions.

Ainsi, pour indexer automatiquement tous les noms propres d'un texte, déclarons la commande suivante:\label{indexauteur}

\begin{latexcode}
\newcommand\auteur[2]{#1~\textsc{#2}\index{#2, #1}\xspace}
\end{latexcode}

\ifdefmacro{\auteur}{
\renewcommand\auteur[2]{#1~\textsc{#2}\xspace}}{\newcommand\auteur[2]{#1~\textsc{#2}\xspace}}

Il suffit ensuite, au cours de la rédaction de son texte, de frapper, par exemple, \cs{auteur}\verb|{Victor}{Hugo}| pour obtenir \enquote{\auteur{Victor}{Hugo}} dans le corps de son texte, indexé sous l'entrée \enquote{Hugo, Victor}.


\subsubsection{Générer l'index}

Si notre fichier principal s'appelle \fichier{exemple.tex},  la commande \cs{makeindex} indique à \LaTeX{}, lors de la compilation avec \XeLaTeX, de créer un fichier \fichier{exemple.idx}, contenant la liste de toutes les entrées. 
\LaTeX{} génère aussi un fichier \fichier{exemple.ilg} qui contient les messages de compilation de l'index, et un fichier \fichier{exemple.ind} qui contient l'index formaté.
L'index  apparaît dans le document à l'emplacement que l'on a indiqué par la commande \csp{printindex}.

\subsection{Allons plus loin}
\subsubsection{Créer des subdivisions}

Il est possible, avec \package{imakeidx},  de créer des subdivisions et des subsubdivisions  pour chaque entrée de l'index. La subdivision se crée de la manière suivante : \begin{english}\cs{index}\verb|{|\meta{entrée}\verb|!|\meta{sous-entrée}\verb|}|\end{english}. La sous-sous-entrée, logiquement, se crée ainsi: \begin{english}\cs{index}\verb|{|\meta{entrée}\verb|!|\meta{sous-entrée}\verb|!|\meta{sous-sous-entrée}\verb|}|\end{english}. On ne peut cependant avoir que ces trois niveaux d'indexation.

On peut ainsi référencer autrement notre premier exemple, en créant les entrées \enquote{Évêques} et \enquote{Ducs} que l'on subdivise:

\begin{latexcode}
Tandis que Charle\index{Charlemagne} était à Rome, il convint 
avec le pape Adrien\index{Adrien} qu’ils enverraient de concert 
des ambassadeurs à Tassilon, duc de Bavière\index{Ducs!Tassilon}.
\textelp{}
Les hommes choisis et envoyés dans cette ambassade furent, de la 
part du pape, les évêques Formose\index{Évêques!Formose}
et Damase\index{Évêques!Damase}\textelp{}.
\end{latexcode}


Supposons que ce texte soit à la page 5; on obtient ainsi dans l'index:
\begin{quotation}
\begin{tabbing}
\hspace{0,5cm}  \= \kill
Adrien, 5 \\
\\
Charlemagne, 5 \\
\\
Ducs \\
\> Tassilon, 5\\
\\
Évêques \\
\> Damase, 5\\
\> Formose, 5
\end{tabbing}
\end{quotation}



Bien entendu, on pourrait encore rajouter une subdivision, en distinguant par exemple \enquote{Clercs} et \enquote{Laïcs}, et dans la première catégorie en distinguant \enquote{Évêques} et \enquote{Papes}. 


 
\subsubsection{Faire des références croisées}

Pour qu'une entrée dans l'index renvoie à une autre entrée, on utilise la commande  \cs{index}\verb|{|\meta{entrée}\verb+|see+\meta{entrée à laquelle on renvoie}\verb|}|. 

Ainsi,  \cs{index}\verb+{Tassilon|see{Ducs}}+ donne dans l'index:

\begin{quotation}
Tassilon, voir Ducs
\end{quotation}
La traduction de \emph{see} change selon la langue indiquée comme \cs{setmainlanguage}\renvoi{french}. 


\begin{attention}
L'utilisation d'un renvoi n'entraîne pas automatiquement l'indexation du terme auquel on renvoie.
Ainsi, il faudrait mettre :
\begin{latexcode}
\index{Tassilon|see{Ducs}}\index{Ducs!Tassilon}
\end{latexcode}
\end{attention}
  

\subsubsection{Créer des entrées sur plusieurs pages}

Si l'on veut référencer dans l'index non pas un mot mais un passage, il faut placer la commande \cs{index}\verb|{|\meta{entrée}\verb+|(}+ au début du passage à indexer et la commande  \cs{index}\verb|{|\meta{entrée}\verb+|)}+ à la fin. Si le passage commence à la page x et se termine à la page y, on obtient dans l'index: 

\begin{quotation}
entrée x-y
\end{quotation}


\subsubsection{Entrées formatées}

L'indexation avec \LaTeX ne gère pas correctement les accents indiqués dans l'argument de la commande \cs{index} : il classe les mots commençant par un accent à la fin de l'index. La syntaxe \cs{index}\verb|{|\meta{entrée}\verb|@|\meta{entrée formatée}\verb+}+ permet de résoudre ce problème. Ainsi, si l'on veut créer une entrée \enquote{écrivains}, qui soit triée à \enquote{e}, il faut insérer \cs{index}\verb|{ecrivains@écrivains}|.

La commande \cs{index}\verb|{|\meta{entrée}\verb|@|\meta{entrée formatée}\verb+}+ permet donc de classer une entrée où l'on veut dans l'index. Pour faire apparaître, par exemple, les empereurs romains dans l'ordre chronologique et non dans l'ordre alphabétique, on peut utiliser les commandes suivantes:

\begin{latexcode}
\index{Empereurs!empereur1@Auguste}
\index{Empereurs!empereur2@Tibère}
\index{Empereurs!empereur3@Claude}
et ainsi de suite. 
\end{latexcode}

On obtient alors :

\begin{quotation}
\begin{tabbing}
\hspace{0,5cm} \= \kill
Empereurs\\
\> Auguste, x\\
\> Tibère, y\\
\> Claude, z 
\end{tabbing}
\end{quotation}

Cette syntaxe est aussi utile pour mettre en évidence une entrée dans l'index en modifiant son aspect : \cs{index}\verb|{|\meta{entrée}\verb|@\textbf{|\meta{entrée formatée}\verb|}}|, fait apparaître l'entrée en gras dans l'index. Ceci est valable pour toutes les commandes agissant sur la fonte. Modifions ainsi comme suit la commande \cs{auteur} que l'on a créée précédemment\renvoi{indexauteur} :

\begin{latexcode}
\newcommand{\auteur[2]}{%
    #1~\textsc{#2}\index{#2 #1@\textsc{#2}, #1}\xspace}
\end{latexcode}

Désormais, mettre \cs{auteur}\verb|{Victor}{Hugo}| produit  dans l'index
 \enquote{\textsc{Hugo},~Victor}.
 
 

\subsubsection{Formater le numéro des pages}

Il peut arriver, lorsqu'une entrée est très souvent représentée dans un texte indexé, que l'on veuille mettre en valeur une de ses occurrences, en faisant apparaître en gras dans l'index le numéro de la page où elle se situe : on utilise alors la commande \cs{index}\verb|{|\meta{entrée}\verb+|textbf}+. 

\begin{attention}
Il s'agit bien de \verb+|textbf+, non de \verb+|\textbf+.
\end{attention}

De même pour faire apparaître le numéro de la page en italique utilise-t-on la commande \cs{index}\verb|{|\meta{entrée}\verb+|textit}+.





\begin{plusloins}
Si vous utilisez le package \package{hyperref}, vous constaterez que celui-ci insère des liens hypertextes vers les pages au sein de l'index. Toutefois si une de ces pages est formatée, le lien disparaît. Nous expliquons sur notre site comment éviter ce problème\footcite{indexhypergras}.
\end{plusloins}


\subsection{Quelques options du package \packagenoidx{imakeidx}}\index[pkg]{imakeidx}



La commande \cs{makeindex} peut recevoir des arguments optionnels sous la forme \cs{makeindex}\oarg{clef$=$valeur} ; s'il y a plusieurs arguments, ils doivent être séparés par une virgule. En voici trois\footnote{Nous en verrons un quatrième pour les index multiples, et un autre lorsque nous évoquerons le formatage de l'index. Pour une liste exhaustive, voir \cite{imakeidx}.} :
\begin{choix}
\item[title] Permet de changer le nom de l'index, qui par défaut est \enquote{Index}. Par exemple, le code suivant permet d'avoir un index intitulé \enquote{Index rerum} 

\begin{latexcode}
\makeindex[title = Index rerum]
\end{latexcode}

\item[columns] Indique le nombre de colonnes. Par défaut, l'index est en deux colonnes. Si l'on veut un index en une seule colonne, il suffit  de mettre  :

\begin{latexcode}
\makeindex[title = Index rerum, columns = 1]
\end{latexcode}

\item[intoc] Indique si l'index doit apparaître ou non dans la table des matières. Par défaut, l'index n'apparaît pas dans la table des matières. Pour l'y mettre, il suffit d'ajouter cet argument dans la liste. 

\begin{latexcode}
\makeindex[title = Index rerum, columns = 1, intoc]
\end{latexcode}
\end{choix}

Il existe une autre commande, la commande \cs{indexsetup}, qui peut, elle aussi, recevoir plusieurs arguments sous la forme
\marg{clef$=$valeur}. Voici ceux qui peuvent être utile lorsque l'on n'a qu'un index (nous en verrons quelques autres à la section suivante):
\begin{choix}
\item[level] Indique le niveau de division (section, chapitre, etc) auquel correspond l'index. Par défaut, il s'agit de \verb|\chapter*|.
\item[toclevel] Indique le niveau de division (section, chapitre, etc) auquel correspond l'index dans la table des matières.  À l'inverse de l'argument précédent, il n'y a pas de contre-oblique :

\begin{latexcode}
\indexsetup[level=\section*, toclevel=section]
\end{latexcode}
\end{choix}


Pour terminer, voici la commande \cs{indexprologue}\oarg{espace}\marg{texte} qui permet de mettre un prologue avant l'index. La commande se place juste avant \cs{printindex}\footnote{Il peut y avoir autant de prologues que d'index.}. L'argument \arg{espace} (par défaut \csp{bigskip}) indique l'espace   entre le prologue de l'index et la première entrée; il doit contenir  une commande d'espacement vertical\renvoi{espace}. L'argument \marg{texte} contient le prologue en lui même. Par exemple, pour un index des notions :

\begin{latexcode}
\indexprologue{Les numéros en gras renvoient aux définitions de notions.}
\printindex
\end{latexcode}


\section{Faire plusieurs index}\label{multiindex}


\subsection{Définir ses index}

Le package \package{imakeidx} permet de faire plusieurs index pour un même document. La première étape consiste à définir ces index. On les déclare dans le préambule en utilisant pour chaque index la commande \csp{makeindex}, déjà étudiée. Chaque index reçoit un nom  abrégé --- à ne pas confondre avec son titre --- qui permettra ensuite d'indiquer  à quel index appartient telle entrée indexée dans votre document. Le nom abrégé de l'index est indiqué comme  option à la commande  \csp{makeindex}.


Ainsi, pour faire un index des noms propres et un index général, on peut déclarer:

\begin{latexcode}
\makeindex[title=Index principal]
\makeindex[name=npr, title=Index des noms propres] 
\end{latexcode}

On remarque ici qu'aucun nom n'a été donné à l'index principal: dans ce cas, le nom abrégé est automatiquement \forme{idx}. À la compilation, seront crées, en plus  des fichiers \fichier{exemple.idx}, \fichier{exemple.ilg} et \fichier{exemple.ind},  les fichiers \fichier{npr.idx}, \fichier{npr.ilg} et \fichier{npr.ind}.


\subsection{Indexer son texte}
Une fois  les index déclarés, il faut passer à l'indexation proprement dite. Le principe est le même que pour un seul index,
mais au lieu de \cs{index}\marg{entrée}, on utilise : 
\csp{index}\oarg{nom abrégé}\marg{entrée}. 

\begin{plusloins}
Toutes les entrées indiquées  par \cs{index}\marg{entrée} sans l'argument \arg{nom  abrégé}, sont automatiquement placées dans l'index général \forme{idx}.
\end{plusloins}

On peut ainsi indexer notre texte d'Éginhard de la façon suivante:

\begin{latexcode}
\index{Charles et la papauté|(}
Tandis que Charle\index[npr]{Charlemagne} était à Rome, il convint
avec le pape Adrien\index[npr]{Adrien} qu’ils enverraient de concert
des ambassadeurs à Tassilon, duc de Bavière\index[npr]{Tassilon}
\textelp{}
Les hommes choisis et envoyés dans cette ambassade furent, de 
la part du pape, les évêques Formose\index[npr]{Formose} et 
Damase\index[npr]{Damase}(…)\index{Charles et la papauté|)}.
\end{latexcode}



\subsection{Imprimer les index}

Pour imprimer un index, il suffit d'utiliser la commande \csp{printindex} en lui passant le nom abrégé de l'index en option. Ainsi, pour imprimer l'index général suivi de celui des noms propres:

\begin{latexcode}
\printindex
\printindex[npr]
\end{latexcode}


\begin{plusloins}
Si l'on veut regrouper tous les index en un seul chapitre dont chaque index est une section, on utilisera   la commande \cs{indexsetup}. On peut lui passer comme option \oarg{noclearpage}, pour  éviter que chaque index commence à une nouvelle page. Bien sûr, chaque index peut avoir son propre prologue. Voici un exemple: 
\begin{latexcode}
\indexsetup{level=\section*,toclevel=section,noclearpage} 
... 
\chapter*{Indices}
\indexprologue{Les numéros en gras renvoient aux définitions de notions.} 
\printindex

\indexprologue{Les auteurs anciens sont indiqués en italiques.}
\printindex[npr] 
\end{latexcode}
\end{plusloins}

\section{Indexer ses sources}

Nous allons maintenant voir comment utiliser les possibilités de \package{biblatex} et de \package{imakeidx} pour établir un index des sources primaires.

Pour comprendre cette section, vous devez vous être familiarisé avec les indications sur les macros bibliographiques\renvoi{macrobiblio}.



\subsection{Premier essai}

La documentation de \package{biblatex}\footcite{biblatex_options} nous informe qu'il existe  au chargement du package une option \option{indexing} qui permet d'indexer automatiquement les références bibliographiques. Comme nous ne souhaitons indexer que les références appelées par les commandes \cs{\meta{prefix}cite}  ---  et non celles appelées par la commande \cs{printbibliography} --- nous attribuons la valeur \option{cite} à cette option. 

\begin{latexcode}
\usepackage[indexing=cite]{biblatex}
\end{latexcode}

Étant donné qu'il faut à la fois interpréter le fichier \ext{bib} et faire un index, nous devons procéder aux compilations dans l'ordre suivant :

\begin{enumerate}
\item Compilation avec \XeLaTeX.
\item Compilation avec Biber.
\item Compilation avec \XeLaTeX pour que les données bibliographiques soient intégrées dans l'index.
\end{enumerate}

On constate cependant deux problèmes : 
\begin{enumerate}
\item La bibliographie se trouve mêlée aux autres entrées de l'index.
\item Plus grave : nous avons des entrées pour les auteurs et des entrées pour les titres, au lieu d'avoir des entrées sous la forme : 
\begin{english}\verb|Auteur!Titre|\end{english}.
\end{enumerate}

En outre nous aimerions :
\begin{enumerate}
\item Limiter l'indexation aux sources primaires.
\item Indexer aussi, comme troisième niveau d'index, le champ \champ{titleaddon} qui nous sert pour les divisions de source\renvoi{divisionsource}. 
\end{enumerate}

\subsection{Création d'un index spécifique}

Pour créer un index spécifique aux sources, rien de particulier : il suffit d'utiliser \package{imakeidx} et la commande \cs{makeindex} :

\begin{latexcode}
\makeindex[name=sources,title=Sources]
\end{latexcode}

\subsection{Modifications des macros de \packagenoidx{biblatex}}\index[pkg]{biblatex}

Nous avons donc notre index spécifique. Mais encore faut-il que nous disions  à \package{biblatex} d'y écrire son index. Pour ce faire nous allons d'abord redéfinir la macro \bibmacro{citeindex} qui est appelée à chaque commande \cs{\meta{prefix}cite}.

\inputminted{exemples/navigation/index-source/citeindex.tex}

\begin{description}
\item[ligne 2] la commande \csp{ifciteindex} vérifie que l'option \option{indexing} de \package{biblatex} est bien égale à \verb|true| ou bien à \verb|cite|: ce qui suit entre accolades est exécuté si tel est le cas.
\item[ligne 3] nous indexons le champ \champ{author}.  Nous utilisons le format d'indexation \verb|sources|.
\item[ligne 4] nous indexons le champ \champ{indextitle}. Ce champ spécial  sert à avoir dans l'index un autre titre que dans le corps du document. Si ce champ est vide \package{biblatex} utilise à la place le champ \champ{title}. Nous utilisons le format d'indexation \verb|sources|.
\item[ligne 5] nous indexons le champ \champ{titleaddon}. Nous utilisons le format d'indexation \verb|sources|.
\item[ligne 6]  une des limitations de \package{biblatex} est qu'il ne peut indexer qu'un seul champ à la fois, et n'est pas capable, pour le moment, de produire des entrées d'index à plusieurs niveaux. 
Avec cette macro \bibmacro{citeindex}, on obtient une indexation séparée pour chaque champ des entrées indexées. 
Or nous voudrions obtenir une indexation correspondant à la commande:
\begin{latexcode} 
\index{auteur!titre!titleaddon}
\end{latexcode}
 C'est pourquoi nous avons conçu un script dans le langage python\footnote{Nous expliquons plus loin comment s'en servir\renvoi{python}.}, qui concatène dans le fichier \ext{idx} les trois indexations en une seule.
Mais avant d'utiliser ce script, il faut  indexer une fausse valeur, la valeur \verb|---|, qui empêchera le script d'indexer tous les champs d'une entrée: nous ne voulons pas obtenir une entrée de la forme : \\
 \verb|\index{author!title!titleaddon!author2!title2!}|
 
Toutefois, ce script python doit être exécuté avant que \LaTeX ne transforme le fichier \ext{idx} en fichier \ext{ind}. Cette transformation est faite automatiquement par le package \package{imakeidx}. Cependant, nous pouvons désactiver cet automatisme pour un index particulier, dans le cas présent pour l'index sources. Pour ce faire, il nous suffit de passer l'option \option{noautomatic} à la commande \cs{makeindex} :\label{noautomatic}

\begin{latexcode}
\makeindex[name=sources,title=Sources,noautomatic]
\end{latexcode}

Il nous faudra alors, après l'exécution du script python, compiler le fichier \fichier{sources.idx} avec le script MakeIndex, afin de produire un fichier \fichier{sources.ind} :
\begin{bashcode}
makeindex sources.idx
\end{bashcode}
\end{description}

Après cela, la compilation avec XeLaTeX affiche correctement l'index.

\subsection{Fomat d'indexation \packagenoidx{biblatex}}\index[pkg]{biblatex}

Nous avons dit que nous utilisions le format d'indexation \verb|sources|. Un format d'indexation \package{biblatex}  est simplement la description de l'opération que \package{biblatex} effectue lorsqu'il doit indexer un champ. 
Il nous faut donc définir ce format grâce aux commandes \csp{DeclareIndexNameFormat} et \csp{DeclareIndexFieldFormat}.

\subsubsection{Indexation des noms}

\begin{english}
\begin{latexcode}
\DeclareIndexNameFormat{sources}{%
  \usebibmacro{index:name}{\index[sources]}{#1}{#3}{#5}{#7}
  }
\end{latexcode}
\end{english}

Nous indiquons en première ligne que nous déclarons un format d'indexation \verb|sources| pour les noms propres. Dans la ligne suivante, nous déclarons ce que nous faisons : nous appelons une macro \bibmacro{index:name}. 

Cette macro est déjà définie par \package{biblatex}. Elle reçoit plusieurs arguments. Le premier argument est la commande à exécuter : ici \cs{index}\verb|[sources]|, qui permet d'indexer dans l'index \verb|sources| défini plus haut. Les autres arguments sont repris des codes de \package{biblatex} et désignent les différentes parties du nom à indexer\footcite[Nous renvoyons le lecteur à la documentation de \package{biblatex} : ][]{biblatex_formats}.


\subsubsection{Indexation des autres champs}

\begin{latexcode}
\DeclareIndexFieldFormat{sources}{%
  \ifcurrentfield{indextitle}{\index[sources]{#1@\emph{#1}}}%
  {\index[sources]{#1}}%
  }
\end{latexcode}

La commande \cs{DeclareIndexFieldFormat} sert à déclarer la manière d'indexer les champs qui ne sont ni des listes ni des noms. La valeur \verb|#1| correspond à la valeur du champ à indexer. En deuxième ligne, nous vérifions grâce à la commande \csp{ifcurrentfield}, que le champ    est \champ{indextitle} : si c'est le cas, nous l'indexons dans l'index \verb|source| en mettant l'emphase sur le titre pour l'affichage final. Sinon, nous l'indexons simplement dans l'index \verb|source|.

\subsection{Compilation et concaténation des index}\label{scriptpython}


Après la  compilation \XeLaTeX, nous obtenons un fichier \fichier{sources.idx}. Si vous l'ouvrez vous constaterez que nous avons des entrées sous la forme : 

\begin{latexcode}
\indexentry{Author}{page}
\indexentry{Titleindex@\emph  {Titleindex}}{page}
\indexentry{Titleaddon}{page}
\indexentry{---}{page}
\end{latexcode}

Nous souhaitons remplacer ces entrées par des entrées sous la forme :

\begin{latexcode}
\indexentry[sources]{Author@Author!Titleindex@
\emph  {Titleindex}!Titleaddon@Titleaddon}{page}
\end{latexcode}

L'auteur de ces lignes a développé un script permettant d'automatiser cette transformation. Par ailleurs ce script modifie également l'ordre de tri pour tenir compte des accents.

Pour utiliser ce script, il vous faut :\label{python}
\begin{itemize}
\item Avoir le logiciel Python installé sur votre ordinateur. Ce logiciel est installé en standard sous Mac Os X et sur la plupart des distributions Linux, mais pas sous Windows\footcite{python_windows}.
\item Télécharger le fichier \url{https://github.com/maieul/indexation-sources/zipball/stable}, le décompresser.
\item Mettre les fichiers \verb|index.py| et \verb|roman.py| dans le répertoire du fichier \ext{idx}. 
\item Ouvrir le fichier \verb|index.py| et modifier la ligne 8 en remplaçant \verb|xxx.idx| par le nom du fichier à concaténer, en l'occurence \verb|sources.idx|.
\item En ligne de commande\renvoi{terminal} se rendre dans le  répertoire, puis frapper l'entrée : \verb|python index.py|.
\end{itemize}


Après cette concaténation nous devons compiler l'index \forme{sources} en ligne de commande, via le script MakeIndex :

\begin{bashcode}
makeindex sources
\end{bashcode}


\subsection{Raffinement}

Nous souhaitons n'indexer que les sources primaires. La solution la plus simple est d'utiliser dans le fichier \ext{bib} un champ personnalisé \champ{usera}. Le package \package{biblatex} permet en effet à l'utilisateur d'utiliser librement un certain nombre de champs\footcite[La liste de ces champs est fournie dans][]{biblatex_custom_fields}.  Dans ce champ, mettre 1 si l'entrée est une source primaire, 2 si l'entrée est une source secondaire.

Il nous suffit de modifier la macro \bibmacro{citeindex}, en introduisant un test (ligne~3) sur la valeur du champ \champ{usera}, grâce à la commande \csp{iffieldequalstr}.
\inputminted{exemples/navigation/index-source/citeindex-usera.tex}

\subsection{Résumé des diverses compilations}

Pour obtenir un index des sources primaires, une fois tous les fichiers mis en place, il nous faut donc procéder dans le terminal aux opérations suivantes :
\begin{enumerate}
\item \verb|xelatex xxx.tex|
\item \verb|biber xxx|
\item \verb|xelatex xxx.tex|\footnote{Si un sommaire se situe en début d'ouvrage, il peut être nécessaire de compiler plusieurs fois.}
\item \verb|python index.py|
\item \verb|makeindex sources|
\item \verb|xelatex xxx|
\end{enumerate}

\begin{plusloins}
Évidemment, il peut être fastidieux de se souvenir de l'ensemble de ces opérations, de les faire et refaire\ldots 

Il existe un programme nommé latexmk qui permet d'automatiser ce genre d'opération : nous en parlons en annexe\renvoi{latexmk}.
\end{plusloins}

